\documentclass[12pt]{report}
\title{System validation document}
\date{\today \\ Version 0.0.1}

\usepackage[latin1]{inputenc}
\usepackage{amsmath}
\usepackage{graphicx}
\usepackage{a4wide}
\usepackage{longtable}
\usepackage{enumitem}
\usepackage{url}
\usepackage{caption}
\usepackage[parfill]{parskip}
\usepackage{appendix}
\usepackage{subcaption}
\usepackage{color}
\usepackage{booktabs}
\usepackage{lineno}
\usepackage{float}
\usepackage{longtable}
\usepackage{colortbl}
\usepackage{amsfonts}
\newcounter{counter}
\usepackage{listings}

%\linenumbers
\modulolinenumbers[5]

\begin{document}
	\begin{titlepage}
		\begin{center}
			\textsc{\Large EINDHOVEN UNIVERSITY OF TECHNOLOGY}\\[1.5cm]
			
			\textsc{\Large System Validation}\\[0.8cm]
			\hrule
			\vspace{0.5cm}
			{ \huge \bfseries EUV Wafer Stepper \\[0.4cm] }
			\hrule
			\vspace{1.5cm}
			\noindent
			\begin{minipage}[t]{0.5\textwidth}
				\begin{flushleft} \large
					\emph{Authors:}\\
					R.M. van den Hurk (0817761)\\
					Z. Ben Snaiba (0748095)\\
					P.M.M. van Wesel (0818131)\\
				\end{flushleft}
			\end{minipage}\\
			\vspace{5cm}
			\begin{minipage}[t][8cm]{0.3\textwidth}
				\begin{flushright} \large
					\today
				\end{flushright}
			\end{minipage}
			
			\vfill
			
		\end{center}
	\end{titlepage}
	
	\tableofcontents
	
	\chapter{System overview}
	
	\section{System overview}
	\begin{figure}
		\centering
		\includegraphics[scale=0.7]{systemmodel}
		\caption{A schematic overview of the system}
		\label{fig:overview}
	\end{figure}
	The Extreme Ultraviolet (EUV) Wafer Stepper is a system developed by ASML to treat silicon wafers for the manufacturing of integrated circuits. Treating a wafer means projecting an image onto a silicon wafer with ultra violet light. Since ultra violet radiation is absorbed by air, this treatment has to happen in a vacuum to ensure the accuracy of the projection. This means wafers will have to enter and exit the system through airlocks, also referred to as sluices. The version of the system described in this document has two sluices. One robot will move wafers from the input to the airlocks or from the airlocks to the output for treated wafers from outside the vacuum chamber. One robot will move wafers from the airlocks to a buffer rack inside the vacuum chamber and vice versa for treated wafers. A last robot, located inside the vacuum chamber, takes wafers from the internal buffer rack to the projector and back once the projector is done. A schematic overview of the system can be found in figure \ref{fig:overview}.
	\section{System components}
	The system can be broken down into individual components to be considered by the software. These components are the following:
	\begin{itemize}
	\item \textbf{Input} - A rack containing untreated wafers located outside of the vacuum chamber.
	\item \textbf{Output} - A rack for treated wafers outside the vacuum chamber.
	\item \textbf{Airlocks} - An array of airlocks providing entrance into the vacuum chamber. Each airlock consists of an inner door, outer door, vacuum pump and air pressure sensor.
	\item \textbf{Buffer Rack} - The buffer rack located inside the vacuum chamber that can temporarily hold wafers.
	\item \textbf{Projector} - The projector inside the vacuum chamber that treats the wafers.
	\item \textbf{Robot A} - The robot that can reach \emph{Input}, \emph{Output} and the \emph{Airlocks}.
	\item \textbf{Robot B} - The robot inside the vacuum chamber that can reach the \emph{Airlocks} and \emph{Buffer Rack}.
	\item \textbf{Robot C} - The other robot inside the vacuum chamber that can reach the \emph{Buffer Rack} and \emph{Projector}.
	\end{itemize}
	
	\chapter{System requirements}
	
	\newcommand{\req}[1]{
		\item[\textbf{R\stepcounter{counter}\arabic{counter}}] {#1}
		\hrule
	}
	
	\newcommand{\reqb}[2]{
		\item[\textbf{{#1}}] {#2}
		\hrule
	}
	
	\section{Liveness requirement}
	\begin{itemize}
		\req{If the system is operating normally, an untreated wafer on the input rack will enter the system, and exit the system after it has been treated.}
	\end{itemize}
	
	\section{Sluice requirements}
	\begin{itemize}
		\req{Only one sluice door can be open at a time.}
		\req{The outer sluice door can only open if there is normal air pressure inside the sluice.}
		\req{The inner sluice door can only open if there is a vacuum inside the sluice.}
		\req{The vacuum pump can only make a vacuum if both sluice doors are closed.}
		\req{The doors of a sluice can not close if a robot is reaching inside.}
	\end{itemize}
	
	\section{Sluice robot requirements}
	\begin{itemize}
		\req{Robot A may only move to a sluice if the outer door of the target sluice is open.}
		\req{Robot B may only move to a sluice if the inside door of the target sluice is open.}
		\req{Robot A\&B may only choose a sluice to put a wafer in that is empty.}
		\req{Robot A\&B may only try to retrieve a wafer from an occupied sluice.}
		\req{Robot A\&B can only put a wafer on an empty spot on the rack.}
		\req{Robot A\&B may only target functioning sluices.}
		\req{Robot A\&B should only try to take wafers from the rack if there are wafers available.}
		\req{Robot B may not access the same spot on the rack as Robot C at the same time.}
	\end{itemize}
	
	\section{Inside robot requirements}
	\begin{itemize}
		\req{Robot C can not access the same spot on the rack as Robot B at the same time.}
		\req{Robot C can not put a wafer on the projection platform when it is occupied.}
		\req{Robot C can not take a wafer from the projection platform if the projection is not done.}
		\req{Robot C can not put a wafer on an occupied place on the rack.}
	\end{itemize}
		
	\section{Projector requirements}
	\begin{itemize}
		\req{The projector only starts its treatment when a wafer is on the projection platform.}
	\end{itemize}
	
	\chapter{System interactions}
\section{System actions}
System actions are actions that the system initiates, so they are an order from the system to subsystems, and can be done at any point.
	\begin{itemize}
\item OpenOuterDoor(N) - Open the outer door of sluice N.
\item CloseOuterDoor(N) - Close the outer door of sluice N.
\item OpenInnerDoor(N) - Open the inner door of sluice N.
\item CloseInnerDoor(N) - Close the inner door of sluice N.
\item PumpVacuum(N) - Make a vacuum in sluice N.
\item ReleaseVacuum(N) - Release the vacuum in sluice N.

\item RobotFromInput( W ) - Robot A picks up wafer W from the input rack.
\item RobotToOutput( W ) - Robot A deposits wafer W on the output rack.

\item RobotToSluice( R, S, W ) - Robot R deposits wafer W it is holding in sluice S.
\item RobotFromSluice( R, S, W ) - Robot R picks up wafer W from sluice S.

\item RobotToRack( R, P, W ) - Robot R deposits wafer W it is holding to position P on the buffer rack.
\item RobotFromRack( R, P, W ) - Robot R takes wafer W from position P on the buffer rack.

\item RobotToProjector( W ) - Robot C deposits wafer W it is holding on the projector.
\item RobotFromProjector( W ) - Robot C picks up wafer W that is currently on the projector.

\item TreatWafer( W ) - The projector treats wafer W.
\end{itemize}

	\section{System events}
	This model uses the notion of system events, which are also just actions. It is assumed however that these
	actions are only allowed when the corresponding event fires. This mean that RobotReset(R) is only allowed for a fraction of time at the moment Robot R fires the event that it has reset. If only one component relies on the event it will be used like any other action. If multiple components rely on the same event instance it will be mapped as a synchronisation communication.
	\begin{itemize}
\item RobotReset( R ) - Robot R has reset to its default position.
\item OuterDoorOpened( S ) - The outer door of sluice S has completely opened.
\item OuterDoorClosed( S ) - The outer door of sluice S has completely closed.
\item InnerDoorOpened( S ) - The inner door of sluice S has completely opened.
\item InnerDoorClosed( S ) - The inner door of sluice S has completely closed.
\item VacuumDone( S ) - The vacuum in sluice S is complete.
\item VacuumReleased( S ) - The vacuum in sluice S is completely released.
\item WaferTreated( W ) - Wafer W is treated by the projector.
	\end{itemize}
	
	\chapter{Requirements in terms of interactions}
	\section{Liveness requirement}
	\begin{itemize}
		\reqb{SR1}{If a wafer appears at position p on the input rack, we should see the following sequence of actions. Other actions can happen between any two of these actions, provided they are not one of those two actions with different parameters.\\
		\begin{itemize}
\item robotTakeWafer( A, p )
\item robotWaferInSluice( A, s ) \emph{- For some arbitrary sluice s}
\item robotWaferFromSluice( B, s )
\item robotDepositWafer( B, p2 ) \emph{- For some arbitrary position p2 on the inner} rack.
\item robotTakeWafer( C, p2 ) \emph{directly followed by} robotCTreatWafer.
\item robotDepositWafer( C, p3 ) \emph{- For some arbitrary position p3 on the inner rack.}
\item robotTakeWafer( B, p3 )
\item robotWaferInSluice( B, s2 ) \emph{- For some arbitrary sluice s2.}
\item robotWaferFromSluice( A, s2 )
\item robotDepositWafer( A, p4 ) \emph{- For some arbitrary position p4 on the output rack.}
\end{itemize}
}
	\end{itemize}
	
	\section{Sluice requirements}
	\begin{itemize}
		\reqb{SR2}{openOuterDoor(N) if innerDoorOpen(N) is true.}
		\reqb{SR3}{openOuterDoor(N) vacuum(N) is false.}
		\reqb{SR4}{openInnerDoor(N) vacuum(N) is true.}
		\reqb{SR5}{pumpVacuum(N) outerDoorOpen(N) and innerDoorOpen(N) are both true.}
		\reqb{SR6}{
\begin{itemize}		
		\item closeOuterDoor(N) can only happen if robotInSluice(N) is false.
\item closeInnerDoor(N) can only happen if robotInSluice(N) is false.
\end{itemize}		
		}
	\end{itemize}
	
	\section{Sluice robot requirements}
	\begin{itemize}
		\reqb{SR7}{robotWaferInSluice(A,N) can only happen if outerDoorOpen(N) is true.}
		\reqb{SR8}{robotWaferInSluice(B,N) can only happen if innerDoorOpen(N) is true.}
		\reqb{SR9}{\begin{itemize}
		\item  robotWaferInSluice(A, N) waferInSluice(N) is false.
		\item robotWaferInSluice(B, N) waferInSluice(N) is false.
		\end{itemize}}
		\reqb{SR10}{
		\begin{itemize}
		\item robotWaferFromSluice(A, N) can only happen if  waferInSluice(N) is true
		\item robotWaferFromSluice(B, N) can only happen if waferInSluice(N) is true.
		\end{itemize}
		}
		\reqb{SR11}{RobotToRack( R, P ) can only happen after another RobotToRack( R, P ) with the same P, if a RobotFromRack( R, P ) has happened with the same P.}
		\reqb{SR12}{openOuterDoor(N), closeOuterDoor(N), openInnerDoor(N), closeInnerDoor(N), pumpVacuum(N), releaseVacuum(N), robotWaferInSluice(R,N),robotWaferFromSluice(R,N),outerDoorOpened(N), outerDoorClosed(N),innerDoorOpened(N),innerDoorClosed(N),vacuumDone(N) and vacuumReleased(N) can only happen if sluiceBroken(N) is false.}
		\reqb{SR13}{RobotFromRack( R, P ) can only happen if IsWaferOnRack( P ) is true for the same P.}
		\reqb{SR14}{RobotTakeWafer(B, p) or RobotDepositWafer(B, p) can only happen if robotAccessingRack(C,C\_Rack) is false.}	\end{itemize}
	
	\section{Inside robot requirements}
	\begin{itemize}
		\reqb{SR15}{RobotTakeWafer(C, p) or RobotDepositWafer(C, p) can only happen if robotAccessingRack(B,C\_Rack) is false. }
		\reqb{SR16}{RobotToProjector( W ) can only happen twice in a row if RobotFromProjector( W ) happens in between.}
		\reqb{SR17}{RobotFromProjector( W ) can only happen if TreatWafer ( W ) happens before.}
		\reqb{SR18}{robotDepositWafer(C, N) can only happen if waferOnRack(C\_Rack, P) is true for some P.}
	\end{itemize}
	\section{Projector requirements}
	\begin{itemize}
		\reqb{SR19}{treatWafer can only happen after robotCTreatWafer if no robotCRetrieve happened after that.}
	\end{itemize}
	
	\chapter{System model}
	\section{Model overview}
	\begin{figure}
		\centering
		\includegraphics[scale=0.7]{schematicoverview}
		\caption{A schematic overview of the software components}
		\label{fig:components}
	\end{figure}
	To verify the system, a model of the software to be made. The software is split up into the following three components:
	\begin{itemize}
	\item \textbf{SluiceManager} - This component manages the sluices, it runs parallel for each sluice. The purpose of this component is to control the individual airlocks.
	\item \textbf{RobotManager} - This component manages the different robots in the system, and runs a parallel process for each robot.
	\item \textbf{ProjectorManager} - The projector is managed by this component.
	\item \textbf{RackManager} - This component's purpose is to monitor the rack.
	\end{itemize}
	
	A schematic view of this system can also be found in figure \ref{fig:components}.
	
	\section{Formal model}
	First the different sorts present in the model are defined.\\
	%Sorts
	\textbf{sort}\\
	\phantom{----} Wafer = \textbf{struct} \emph{wafer}$( id:\mathbb{Z}, treated:\mathbb{B} )$;\\
	\phantom{----} WaferSet = Set( Wafer );\\
	\phantom{----} WaferList = List( Wafer );\\
	\phantom{----} RobotID = \textbf{struct} $R_a | R_b | R_c$;\\
	\phantom{----} SluiceID = \textbf{struct} $S_0 | S_1$;\\
	\phantom{----} SluiceDoorState = \textbf{struct} $inner\_open|closed|outer\_open$;\\
	\\
	Some global parameters can be defined that hold for the system.\\
	%Params
	\textbf{eqn}\\
	\phantom{----} RackSize = 2;\\
	\phantom{----} NumWafers = $X$;\\
	\phantom{----} NumSluices = 2;\\
	\phantom{----} Dummy = \emph{wafer}$( -1, false)$;\\
	\\
	Some helper-equations are defined, the names are self-explanatory.\\
	%Helper equations
	\textbf{eqn}\\
	\phantom{----} IsTreated( \emph{wafer}$( w, b )$ ) = $b$;\hfill for all $w \in \mathbb{Z}, b \in \mathbb{B}$\\
	\phantom{----} IsDummy(  \emph{wafer}$( w, b )$ ) = $w == -1$;\hfill for all $w \in \mathbb{Z}, b \in \mathbb{B}$\\
	\phantom{----} OnRack$( p ) = 0 \leq p < \text{RackSize}$;\hfill for all $p \in \mathbb{N}$\\
	\phantom{----} TreatedWafer(  \emph{wafer}$( w, b )$ ) =  \emph{wafer}$( w, true )$;\hfill for all $w \in \mathbb{Z}, b \in \mathbb{B}$\\
	\\
	\phantom{----} \emph{for all $l \in \text{WaferList}, p \in \mathbb{N}, w \in \text{Wafer}$, where OnRack($p$) = true}:\\
	\phantom{----} PutInList$( l, p, w )$ = [];\hfill if $l=[]$\\
	\phantom{----} PutInList$( l, p, w )$ = $w$ $|>$ PutInList$( tail( l ), p, w )$;\hfill if $0 < \#l \land \#l = \text{RackSize} - p$\\
	\phantom{----} PutInList$( l, p, w )$ = $ head( l )$ $|>$ PutInList$( tail( l ), p, w )$;\hfill if $0 < \#l \land \#l \neq \text{RackSize} - p$\\
	\\
	\phantom{----} InitRack( 0 ) = [];\\
	\phantom{----} InitRack( n ) = InitRack( n - 1 ) $<|$ Dummy;\hfill for all $n \in \mathbb{N}$\\
	\\
	\phantom{----} InitWaferList( 0 ) = [];\\
	\phantom{----} InitWaferList( n ) = InitWaferList( n - 1 ) $<|$ \emph{wafer}($n$, \emph{false});\hfill for all $n \in \mathbb{N}$\\
	\\
	The main part of the system model are the processes that make up the system. There is one process for each of the four components in the system. Multiple instances of each process can be run in parallel with different parameters, as explained at the initialisation.\\
	\\%Processes
	The first process is the robot process, consisting of three parts, one for each robot. Each model waits for its RobotReset event after each action before continuing. This event might require a synchronisation step if two components are waiting for the same event.\\
	If robot A is holding a treated wafer it will put it down on the output. If it is holding an untreated wafer, it will try to put it in a sluice and increase its counter. If robot A is not holding anything, it will either take a wafer from one of the sluices if there is one available and decrease its counter, or if the input list is not empty, and the robot's counter is less than the number of sluices, it will pick up a new wafer from the input. The purpose of the counter is to prevent the robot from picking up a wafer when there are no sluices available to put it down in.\\
	If robot B is holding a treated wafer it will try to put it in one of the sluices. If it is holding an untreated wafer, it will put it down on the rack. If robot B is not holding a wafer, it will try to take an untreated wafer from the sluices, or take a treated wafer from the rack.\\
	Robot C will put a wafer on the projector if it is holding an untreated wafer. It will then wait until the projector is done before picking up the wafer again and putting it back on the rack. If robot C is not holding anything and is not waiting for the projector it will try to pick up a wafer from the rack.\\
	{\small
	%Robot
	\textbf{proc}\\
	\phantom{---} R( \emph{rID}:RobotID, $occupied$:$\mathbb{B}$, $w$:Wafer, $in$:WaferList, $out$:WaferSet, $c$:$\mathbb{N}$ ) =\\
	\phantom{-------} \emph{rID} = $R_a \rightarrow$\\
	\phantom{----------} $occupied$ $\rightarrow$\\
	\phantom{--------------} IsTreated($w$)$\rightarrow$RobotToOutput($w$)$\cdot$RobotReset(\emph{rID})$\cdot$R(\emph{rID},$false$,Dummy,$in$,$out$+$\{w\}$,$c$)\\
	\phantom{--------------} $\Diamond \sum\nolimits_{s \in \text{SluiceID}}\cdot$R2S\_R(\emph{rID},$s$,$w$)$\cdot$RReset\_O(\emph{rID})$\cdot$R(\emph{rID},$false$, Dummy,$in$,$out$,$c$+$1$)\\
	\phantom{----------} $\Diamond$\\
	\phantom{--------------} $in \neq [] \land c < NumSluices \rightarrow$ ( RobotFromInput( head($in$) )$\cdot$\\
	\phantom{---------------------} RobotReset(\emph{rID}))$\cdot$R(\emph{rID},$true$,head($in$),tail($in$),$out$,$c$) ) \\
	\phantom{--------------} + $\sum\nolimits_{w_2 \in \text{Wafer}, s \in \text{SluiceID}}$S2R\_R(\emph{rID},$s$,$w_2$)$\cdot$RReset\_D(\emph{rID})$\cdot$R(\emph{rID}, $true$,$w_2$,$in$,$out$,$c$-$1$)\\
	\\
	\phantom{-------} + \emph{rID} = $R_b \rightarrow$\\
	\phantom{----------} $occupied$ $\rightarrow$\\
	\phantom{--------------} IsTreated($w$)$\rightarrow\sum\nolimits_{s\in\text{SluiceID}}$R2S\_R(\emph{rID},$s$,$w$)$\cdot$RReset\_O(\emph{rID})$\cdot$R(\emph{rID},\emph{false},Dummy,$in$,$out$,$c$)\\
	\phantom{--------------} $\Diamond \sum\nolimits_{p\in \mathbb{N}}$OnRack($p$)$\rightarrow$R2Rack\_R(\emph{rID},$p$,$w$)$\cdot$RReset\_O(\emph{rID})$\cdot$R(\emph{rID},\emph{false},Dummy,$in$,$out$,$c$)
	\phantom{----------}  $\Diamond$\\
	\phantom{--------------} $\sum\nolimits_{w_2 \in \text{Wafer}, s \in \text{SluiceID}}$S2R\_R(\emph{rID},$s$,$w_2$)$\cdot$RReset\_D(\emph{rID})$\cdot$R(rID,\emph{true},$w_2$,$in$,$out$,$c$)\\
	\phantom{--------------} +$\sum\nolimits_{w_2 \in \text{Wafer}, p \in \mathbb{N}}$IsTreated($w_2$)$\land$OnRack($p$)$\rightarrow$Rack2R\_R(\emph{rID},$p$,$w_2$)$\cdot$\\
	\phantom{-----------------------------} RReset(\emph{rID})$\cdot$R(\emph{rID},\emph{true},$w_2$,$in$,$out$,$c$)\\
	\\
	\phantom{-------} + \emph{rID} = $R_c \rightarrow$\\
	\phantom{----------} $occupied$ $\rightarrow$\\
	\phantom{--------------} R2P\_R($w$)$\cdot$RReset\_O(\emph{rID})$\cdot$P2R\_R(TreatedWafer($w$))$\cdot$RReset\_D(\emph{rID})$\cdot\sum\nolimits_{p \in \mathbb{N}}$OnRack($p$)$\rightarrow$\\
	\phantom{------------------} R2Rack\_R(\emph{rID},$p$,TreatedWafer($w$))$\cdot$RReset\_O(\emph{rID})$\cdot$R(\emph{rID},\emph{false},Dummy,$in$,$out$,$c$)\\
	\phantom{----------} $\Diamond$\\
	\phantom{--------------} +$\sum\nolimits_{w_2 \in \text{Wafer}, p \in \mathbb{N}}\neg$IsTreated($w_2$)$\land$OnRack($p$)$\rightarrow$Rack2R\_R(\emph{rID},$p$,$w_2$)$\cdot$\\
	\phantom{-----------------------------} RReset\_D(\emph{rID})$\cdot$R(\emph{rID},\emph{true},$w_2$,$in$,$out$,$c$);
	}\\
	\\
	The sluice process holds parameters for its id, if it is occupied or not, the wafer that is in that wafer, the state of the door and if the sluice is broken or not. Sluices will wait until robots have returned to their rest position before closing their doors. If there is no wafer in the sluice and the inner/outer door is open, the sluice will accept a wafer from robot B/robot A respectively and start closing. If there is a wafer in the sluice and the inner/outer door is open, the sluice will wait until robot B/robot A respectively takes the wafer out of it. If the doors of a sluice are closed and the wafer inside is treated, the sluice will release the vacuum. Once the vacuum is released it will start opening the outer doors. If the sluice doors are closed and there is an untreated wafer inside, the sluice will pumping a vacuum. If the vacuum is done the sluice will open its inner doors. On top of this the wafer could non-deterministically break at any point.\\
	Note: This protocol for the sluices means that there can never be more wafers in the vacuum chamber than the number of sluices. Therefore the spaces on the rack could be limited to the number of sluices in the system.\\
	%Sluice
	{\small
	\textbf{proc}\\
	\phantom{---} S( $id$:SluiceID, $occupied$:$\mathbb{B}$, $w$:Wafer, $door$:DoorState, $broken$:$\mathbb{B}$ ) =\\
	\phantom{------} $\neg broken \rightarrow$\\
	\phantom{---------} $\neg occupied \rightarrow$\\
	\phantom{-------------} $door=outer\_open \rightarrow\sum\nolimits_{w_2\in \text{Wafer}}$R2S\_S($R_a$,$id$,$w_2$)$\cdot$RReset\_D($R_a$)$\cdot$CloseOuterDoor($id$)$\cdot$\\
	\phantom{-----------------} OuterDoorClosed($id$)$\cdot$PumpVacuum($id$)$\cdot$VacuumDone($id$)$\cdot$S($id$,\emph{true},$w_2$,$closed$,\emph{false})\\
	\phantom{-------------} + $door=inner\_open \rightarrow\sum\nolimits_{w_2\in \text{Wafer}}$R2S\_S($R_b$,$id$,$w_2$)$\cdot$RReset\_D($R_b$)$\cdot$CloseInnerDoor($id$)$\cdot$\\
	\phantom{-----------------} InnerDoorClosed($id$)$\cdot$ReleaseVacuum($id$)$\cdot$VacuumReleased($id$)$\cdot$S($id$,\emph{true},$w_2$,$closed$,\emph{false})\\
	\phantom{---------} $\Diamond$\\
	\phantom{-------------} $door=outer\_open \rightarrow$S2R\_S($R_a$,$id$,$w$)$\cdot$RReset\_O($R_a$)$\cdot$S($id$,\emph{false},Dummy,$door$,\emph{false})\\
	\phantom{-------------} + $door=inner\_open \rightarrow$S2R\_S($R_b$,$id$,$w$)$\cdot$RReset\_O($R_b$)$\cdot$S($id$,\emph{false},Dummy,$door$,\emph{false})\\
	\phantom{---------} + $door=closed \land \text{IsTreated}(w) \rightarrow$\\
	\phantom{-------------} OpenOuterDoor($id$) $\cdot$ OuterDoorOpened($id$) $\cdot$ S($id$, $occupied$, $w$, $outer\_open$, \emph{false})\\
	\phantom{---------} + $door=closed \land \neg\text{IsTreated}(w) \rightarrow$\\
	\phantom{-------------} OpenInnerDoor($id$) $\cdot$ InnerDoorOpened($id$) $\cdot$ S($id$, $occupied$, $w$, $inner\_open$, \emph{false})\\
	\phantom{---------} + SluiceBroken($id$) $\cdot$ S($id$, $occupied$, $w$, $door$, \emph{true});\\
	}\\
	The projector process keeps track of if there is a wafer inside, and which wafer that is. If the projector is occupied, it will treat the wafer and once that is done wait until the wafer was picked up by robot C before going back to its unoccupied state. If there is no wafer on the projector it will accept a wafer from robot C.\\
	%Projector
	{\small
	\textbf{proc}\\
	\phantom{---} P( $occupied$:$\mathbb{B}$, $w$:Wafer ) =\\
	\phantom{------} $occupied\rightarrow$\\
	\phantom{---------} TreatWafer($w$)$\cdot$WaferTreated($w$)$\cdot$P2R\_P(TreatedWafer($w$))$\cdot$RReset\_O($R_c$)$\cdot$P(\emph{false},Dummy)\\
	\phantom{------} $\Diamond$\\
	\phantom{---------} $\sum\nolimits_{w_2\in \text{Wafer}}$ R2P\_P($w_2$) $\cdot$ RReset\_D($R_c$) $\cdot$ P(\emph{true}, $w_2$);
	}\\
	The rack process is just to \\
	%Rack
	{\small
	\textbf{proc}\\
	\phantom{---} Rack( $wl$:WaferList ) =\\
	\phantom{-------} $\sum\nolimits_{w\in \text{Wafer}, p \in \mathbb{N}, r \in \text{RobotID}}$ OnRack($p$) $\rightarrow$ IsDummy( $wl.p$ ) $\rightarrow$\\
	\phantom{-----------} R2Rack\_Rack($r$, $p$, $w$) $\cdot$ RReset\_D($r$) $\cdot$ Rack(PutInList($wl$, $p$, $w$))\\
	\phantom{-------} + $\sum\nolimits_{p \in \mathbb{N}, r \in \text{RobotID}}$ OnRack( $p$ ) $\rightarrow \neg$IsDummy($wl.p$)$\rightarrow$\\
	\phantom{-----------} Rack2R\_Rack($r$, $p$, $wl.p$) $\cdot$ RReset\_O( $r$ ) $\cdot$ Rack(PutInList($wl$, $p$, Dummy));
	}\\

	The initialisation of the system without any hiding and all steps allowed can be found below. Note the RobotReset synchronisation of steps RReset\_O(rigin) and RReset\_D(estination), which is for when two components are waiting on the same RobotReset event. An instance of the robot process is started for each process. Robot A recievs the input list, since robot A is the only robot that can reach the input. This is the entry point for wafers into the system.\\
	A process is initialised for the rack with a list of dummies equal to the size of the rack.\\
	For each sluice a parallel instance of the sluice process is started. Each sluice is empty initially and has the outer doors open.\\
	A process is started for the projector that is empty at the start.\\
	%Initialisation\\
	\textbf{init}\\
	$\nabla_{\text{ \{RobotToSluice, RobotFromSluice, RobotToProjector, RobotFromProjector, RobotToRack, RobotFromRack,}}$\\
	\phantom{--} $_{ \text{ RobotFromInput, RobotFromOutput,}}$\\
	\phantom{--} $_{ \text{ CloseOuterDoor, CloseInnerDoor, OpenInnerDoor, OpenOuterDoor, PumpVacuum, ReleaseVacuum,}}$\\
	\phantom{--} $_{ \text{ RobotReset, WaferTreated, OuterDrooOpened, OuterDoorClosed, InnerDoorOpened, InnerDoorClosed, VacuumDone, VacuumReleased, SluiceBroken\}}}$(\\
	\phantom{---} $\Gamma_{\{R2S\_R|R2S\_S\rightarrow RobotToSluice,S2R\_S|S2R\_R\rightarrow RobotFromSluice,}$\\
	\phantom{-----} $_{ R2P\_R|R2P\_P\rightarrow RobotToProjector, P2R\_P|P2R\_R\rightarrow RobotFromProjector,}$\\
	\phantom{-----} $_{ R2Rack\_R|R2Rack\_P\rightarrow RobotToRack, Rack2R\_P|Rack2R\_R\rightarrow RobotFromRack,}$\\
	\phantom{-----} $_{ RReset\_O|RReset\_D\rightarrow RobotReset \}}$(\\
	\phantom{---------} R( $R_a$, \emph{false}, $Dummy$, \emph{InitWaferList(NumWafers)}, $\emptyset$, 0 ) $||$\\
	\phantom{---------} R( $R_b$, \emph{false}, $Dummy$, [], $\emptyset$, 0 ) $||$\\
	\phantom{---------} R( $R_c$, \emph{false}, $Dummy$, [], $\emptyset$, 0 ) $||$\\
	\phantom{---------} Rack( $InitRack(RackSize)$ ) $||$\\
	\phantom{---------} S( $S_0$, \emph{false}, $Dummy$, $outer\_open$, \emph{false} ) $||$\\
	\phantom{---------} S( $S_1$, \emph{false}, $Dummy$, $outer\_open$, \emph{false} ) $||$\\
	\phantom{---------} P( \emph{false}, $Dummy$ )\\
	\phantom{---} )\\
	);
	
	\chapter{Verification}

    \section{Requirement verification}
    
    \textbf{Traceability matrix}\\
    \begin{tabular}{| l | l | l | }
        \hline
        Requirement ID & System Requirement ID & Modal formula ID \\ \hline
        R1 & SR1 & MCF1 \\ \hline 
        R2 & SR2 & MCF2inner, MCF2outer \\ \hline 
        R3 & SR3 & MCF3 \\ \hline 
        R4 & SR4 & MCF4 \\ \hline 
        R5 & SR5 & MCF5 \\ \hline 
        R6 & SR6 & MCF6a1, MCF6a2, MCF6b1, MCF6b2 \\ \hline 
        R7 & SR7 & MCF71, MCF72 \\ \hline 
        R8 & SR8 & MCF82, MCF82 \\ \hline 
        R9 & SR9 & MCF9a, MCF9b \\ \hline 
        R10 & SR10 & MCF10a, MCF10b \\ \hline 
        R11 & SR11 & MCF11 \\ \hline 
        R12 & SR12 & MCF12 \\ \hline 
        R13 & SR13 & MCF13b, MCF13c \\ \hline 
        R14 & SR14 & MCF14 \\ \hline 
        R15 & SR15 & MCF15 \\ \hline 
        R16 & SR16 & MCF16 \\ \hline 
        R17 & SR17 & MCF17 \\ \hline 
        R18 & SR18 & MCF18 \\ \hline 
        R19 & SR19 & MCF19 \\  \hline 
    \end{tabular}
    
    \pagebreak
    
    \textbf{Modal $\mu$-calculus formulas}\\
    \begin{longtable}{p{\textwidth}}
        \textbf{MCF1}\\
        $[true^{\star}] \forall w \in Wafer. [TreatWafer(w).\overline{RobotToOutput(w)}^{\star}.RobotToOutput(w)]false$\\
        \hline

        \textbf{MCF2inner}\\
        $\forall s \in SluiceID.[true^{\star}.InnerDoorOpened(s).\overline{InnerDoorClosed(s)}^{\star}.OpenOuterDoor(s)]false$\\
        \hline

        \textbf{MCF2outer}\\
        $\forall s \in SluiceID.[true^{\star}.OuterDoorOpened(s).\overline{OuterDoorClosed(s)}^{\star}.OpenInnerDoor(s)]false$\\
        \hline 
        
        \textbf{MCF3}\\
        $[true^{\star}] \forall s \in SluiceID. [PumpVacuum(s).\overline{ReleaseVacuum}^{\star}.OpenOuterDoor(s)]true$\\
        \hline
        
        \textbf{MCF4}\\
        $\forall s \in SluiceID.[true^{\star}.ReleaseVacuum(s).\overline{PumpVacuum}^{\star}.OpenInnerDoor(s)]true$\\
        \hline
        
        \textbf{MCF5}\\
        $\forall s \in SluiceID.[true^{\star}.(OpenInnerDoor(s).\overline{InnerDoorClose(s)}^{\star}.PumpVacuum(s) \vee OpenOuterDoor(s).\overline{OpenOuterDoorClose(s)}^{\star}.PumpVacuum(s))]false$ \\
        \hline
        
        \textbf{MCF6a1}\\
        $[true^{\star}] \forall w \in Wafer, s \in SluiceID.[RobotFromSluice(Ra,s,w).\overline{RobotReset(Ra)}$ \\
        $.CloseOuterDoor(s)]false$\\
        \hline
        
        \textbf{MCF6a2}\\
        $[true^{\star}] \forall w \in Wafer, s \in SluiceID.[RobotToSluice(Ra,s,w).\overline{RobotReset(Ra)}$ \\
        $.CloseOuterDoor(s)]false$\\
        \hline
        
        \textbf{MCF6b1}\\
        $[true^{\star}] \forall w \in Wafer, s \in SluiceID.[RobotFromSluice(Rb,s,w).\overline{RobotReset(Rb)}$ \\
        $.CloseOuterDoor(s)]false$\\
        \hline
        
        \textbf{MCF6b2}\\
        $[true^{\star}] \forall w \in Wafer, s \in SluiceID.[RobotToSluice(Rb,s,w).\overline{RobotReset(Rb)}$ \\
        $.CloseOuterDoor(s)]false$\\
        \hline
        
        \textbf{MCF71}\\
        $[true^{\star}] \forall w \in Wafer, s \in SluiceID.[OuterDoorClosed(s).\overline{OuterDoorOpened}^{\star}$ \\ 
        $.RobotToSluice(Ra,s,w)]false$\\
        \hline
        
        \textbf{MCF72}\\
        $[true^{\star}] \forall w \in Wafer, s \in SluiceID.[OuterDoorClosed(s).\overline{OuterDoorOpened}^{\star}$ \\
        $.RobotFromSluice(Ra,s,w)]false$\\
        \hline 
        
        \textbf{MCF81}\\
        $[true^{\star}] \forall w \in Wafer, s \in SluiceID.[InnerDoorClosed(s).\overline{InnerDoorOpened(s)}^{\star}$ \\
        $.RobotToSluice(Rb, s, w)]false$\\
        \hline  
        
        \textbf{MCF82}\\
        $[true^{\star}] \forall w \in Wafer, s \in SluiceID.[InnerDoorClosed(s).\overline{InnerDoorOpened(s)}^{\star}$ \\
        $.RobotFromSluice(Rb, s, w)]false$\\
        \hline     
        
        \textbf{MCF9a}\\
        $[true^{\star}] \forall w1,w2 \in Wafer, s in SluiceID.[RobotToSluice(Ra,s,w1).\overline{RobotFromSluice(Rb,s,w1)}^{\star}$ \\
        $.RobotToSluice(Ra,s,w2)]false$\\
        \hline
        
        \textbf{MCF9b}\\
        $[true^{\star}] \forall w1,w2 \in Wafer, s in SluiceID.[RobotToSluice(Rb,s,w1).\overline{RobotFromSluice(Ra,s,w1)}^{\star}$ \\
        $.RobotToSluice(Rb,s,w2)]false$ \\
        \hline
        
        \textbf{MCF10a}\\
        $[true^{\star}] \forall w \in Wafer, s \in SluiceID. [\overline{RobotToSluice(Rb,s,w)}^{\star}.RobotFromSluice(Ra,s,w)]$ \\
        $false$ \\
        \hline
        
        \textbf{MCF10b}\\
        $[true^{\star}] \forall w \in Wafer, s \in SluiceID. [\overline{RobotToSluice(Ra,s,w)}^{\star}.RobotFromSluice(Rb,s,w)]$ \\
        $false$ \\
        \hline
        
        \textbf{MCF11}\\
        $[true^{\star}] \forall p \in \mathbb{N}, r \in RobotID, w \in Wafer. [RobotToRack(r,p,w).\overline{RobotFromRack(r,p,w)}^\star$ \\
        $.RobotToRack(r,p,w)]false$ \\
        \hline
        
        \textbf{MCF12} \\
        $[true^{\star}] \forall s \in SluiceID [SluiceBroken(n).true^{\star}.(CloseInnerDoor(s) \vee CloseOuterDoor(s) \vee OpenInnerDoor(s) \vee OpenOuterDoor(s) \vee PumpVacuum(s) \vee ReleaseVacuum(s) \vee SluiceBroken(s))]false$\\
        \hline
        
        \textbf{MCF13b} \\
        $[true^{\star}] \forall p \in \mathbb{N}, w \in Wafer.[\overline{RobotToRack(Rb,p,w}^{\star}.RobotFromRack(Rc, p, w)]false$ \\
        \hline
        
        \textbf{MCF13c} \\
        $[true^{\star}] \forall p \in \mathbb{N}, w \in Wafer.[\overline{RobotToRack(Rc,p,w}^{\star}.RobotFromRack(Rb, p, w)]false$ \\
        \hline
        
        \textbf{MCF14} \\
        $[true^{\star}] \forall p \in \mathbb{N}, w \in Wafer. [(RobotToRack(Rc,p,w) \vee RobotFromRack(Rc,p,w)).\overline{RobotReset(Rc}^{\star}.(RobotRack(Rb,p,w) \vee RobotFromRack(Rb,p,w)]false$ \\
        \hline
        
        \textbf{MCF15} \\
        $[true^{\star}] \forall p \in \mathbb{N}, w \in Wafer. [(RobotToRack(Rb,p,w) \vee RobotFromRack(Rb,p,w)).\overline{RobotReset(Rb}^{\star}.(RobotRack(Rc,p,w) \vee RobotFromRack(Rc,p,w)]false$ \\
        \hline
        
        \textbf{MCF16} \\
        $[true^{\star}] \forall w,w2 \in Wafer. [RobotToProjector(w).\overline{RobotFromProjector(w)}^{\star}$ \\
        $.RobotToProjector(w2)]false$ \\
        \hline
        
        \textbf{MCF17} \\
        $[true^{\star}] \forall w \in Wafer. [RobotToProjector(w).\overline{TreatWafer(W)}^{\star}.RobotFromProjector(w)]false$ \\
        \hline
        
        \textbf{MCF18} \\
        $[true^{\star}] \forall w \in Wafer, p \in \mathbb{N}. [RobotToRack(Rc, p, w).\overline{RobotFromRack(Rb,p,w)}^{\star}$ \\
        $.RobotToRack(Rc,p,w)]false$ \\
        \hline
        
        \textbf{MCF19} \\
        $[true^{\star}] \forall w \in Wafer. [RobotToProjector(w).true^{\star}.RobotFromProjector(w).true^{\star}$ \\
        $.TreatWafer(w)]false$ \\
        \hline
    \end{longtable}
	
	\section{Verification procedure}
	Several steps were taken to verify the correctness of this model. First, the model was translated into mCRL2 syntax. Next, all modal formulas were translated into the mcf format used by the mCRL2 tools. Once both translations were done, a linear process specification was generated from the mCRL2 input. This LPS was used to verify the requirements by generating a parameterized boolean expression system for each modal formula.\\
	For the verification of the system release version 201409.1.13218 of the mCRL2 tools was used.\\
	\\
	The system was verified for wafer counts from 1 to 5 to avoid any statespace problems. This should nonetheless present an accurate view of the system, as the maximum amount of wafers inside the vacuum chamber is at most the number of sluices, which is two.
	
	\section{Verification results}
	Results of the verification: matrix numwafers x requirement, value = true/false.	
	
	\chapter{Appendix A}

	\lstinputlisting[breaklines=true]{mcrl2-model/euv_waferstepper.mcrl2}
	\chapter{Appendix B}
	Modal formulas in mcrl format

\end{document} 